%%%%%%%%%%%%%%%%%%%%%%%%%%%%%%%%%%%%%%%%%%%%%%%%%%%%%%%%%%%%%%%%%%%%%%%%%%%%%%%%
%% Ein Header für LaTeX-Dokumente, damit man Dinge immer wieder konsequent    %%
%% gleich schreibt. In allen Dokumenten eingebunden per                       %%
%%   %%%%%%%%%%%%%%%%%%%%%%%%%%%%%%%%%%%%%%%%%%%%%%%%%%%%%%%%%%%%%%%%%%%%%%%%%%%%%%%%
%% Ein Header für LaTeX-Dokumente, damit man Dinge immer wieder konsequent    %%
%% gleich schreibt. In allen Dokumenten eingebunden per                       %%
%%   %%%%%%%%%%%%%%%%%%%%%%%%%%%%%%%%%%%%%%%%%%%%%%%%%%%%%%%%%%%%%%%%%%%%%%%%%%%%%%%%
%% Ein Header für LaTeX-Dokumente, damit man Dinge immer wieder konsequent    %%
%% gleich schreibt. In allen Dokumenten eingebunden per                       %%
%%   %%%%%%%%%%%%%%%%%%%%%%%%%%%%%%%%%%%%%%%%%%%%%%%%%%%%%%%%%%%%%%%%%%%%%%%%%%%%%%%%
%% Ein Header für LaTeX-Dokumente, damit man Dinge immer wieder konsequent    %%
%% gleich schreibt. In allen Dokumenten eingebunden per                       %%
%%   \input{/home/bielefeldt/.local/Abbrevations.tex}                         %%
%%                                                                            %%
%% (C) 2018 Philipp Bielefeldt                                                %%
%%%%%%%%%%%%%%%%%%%%%%%%%%%%%%%%%%%%%%%%%%%%%%%%%%%%%%%%%%%%%%%%%%%%%%%%%%%%%%%%

%% Normale Referenzen
\newcommand{\xdir}{\textit{X}-direction}
\newcommand{\ydir}{\textit{Y}-direction}
\newcommand{\zdir}{\textit{Z}-direction}
\newcommand{\xaxis}{\textit{X}-axis}
\newcommand{\yaxis}{\textit{Y}-axis}
\newcommand{\zaxis}{\textit{Z}-axis}
\newcommand{\xyplane}{\textit{XY}-plane}
\newcommand{\xzplane}{\textit{XZ}-plane}
\newcommand{\yzplane}{\textit{YZ}-plane}

%% Programme & Tools
\newcommand{\gfi}{\texttt{G\small{ENFIT}}\,\uppercase\expandafter{\romannumeral 1}}
\newcommand{\gfii}{\texttt{G\small{ENFIT}}\,\uppercase\expandafter{\romannumeral 2}}
\newcommand{\explora}{\texttt{ExPlORA}}
\newcommand{\geant}{\texttt{Geant}}
\newcommand{\minuit}{\texttt{M\small{INUIT}}}

\newcommand{\kalman}{K\'alm\'an}

%% Experimente
\newcommand{\fopi}{\textit{FOPI}}
\newcommand{\elsa}{\textit{ELSA}}
\newcommand{\cbet}{\texttt{CBELSA/TAPS}}

%%%%%%%%%%%%%%%%%%%%%%%%%%%%%%%%%%%%%%%%%%%%%%%%%%%%%%%%%%%%%%%%%%%%%%%%%%%%%%%%
%% Sonstiges
\newcommand{\ra}{\ensuremath{\Rightarrow}}
\newcommand{\treg}{$^\text{\tiny{\textregistered}}$ }

% Checkboxes
\newcommand{\cbd}{\item[\CheckedBox]}
\newcommand{\cbt}{\item[\Square]}
\newcommand{\cbo}{\item[]}

%% Dingbats
\usepackage{pifont}% http://ctan.org/pkg/pifont
\newcommand{\cmark}{\ding{52}}
\newcommand{\xmark}{\ding{55}}
% http://willbenton.com/wb-images/pifont.pdf 
% https://tex.stackexchange.com/questions/42619/x-mark-to-match-checkmark

%% Bildunterschriften
\newcommand{\mc}[1]{\caption*{\tiny{\color{gray} #1 }}}

%% Standart-Bilder in Beamer
%%  Erzeugt eine column mit Breite #1 (vorher unbedingt \begin{colunms} setzen!)
%%  darin das Bild mit Namen #2 und die Unterschrift #3
\newcommand{\bfig}[3]{
  \column{#1\textwidth}
  \centering
    \begin{figure}
      \includegraphics[width=0.95\textwidth]{#2}
      \mc{#3}
    \end{figure}
}

                         %%
%%                                                                            %%
%% (C) 2018 Philipp Bielefeldt                                                %%
%%%%%%%%%%%%%%%%%%%%%%%%%%%%%%%%%%%%%%%%%%%%%%%%%%%%%%%%%%%%%%%%%%%%%%%%%%%%%%%%

%% Normale Referenzen
\newcommand{\xdir}{\textit{X}-direction}
\newcommand{\ydir}{\textit{Y}-direction}
\newcommand{\zdir}{\textit{Z}-direction}
\newcommand{\xaxis}{\textit{X}-axis}
\newcommand{\yaxis}{\textit{Y}-axis}
\newcommand{\zaxis}{\textit{Z}-axis}
\newcommand{\xyplane}{\textit{XY}-plane}
\newcommand{\xzplane}{\textit{XZ}-plane}
\newcommand{\yzplane}{\textit{YZ}-plane}

%% Programme & Tools
\newcommand{\gfi}{\texttt{G\small{ENFIT}}\,\uppercase\expandafter{\romannumeral 1}}
\newcommand{\gfii}{\texttt{G\small{ENFIT}}\,\uppercase\expandafter{\romannumeral 2}}
\newcommand{\explora}{\texttt{ExPlORA}}
\newcommand{\geant}{\texttt{Geant}}
\newcommand{\minuit}{\texttt{M\small{INUIT}}}

\newcommand{\kalman}{K\'alm\'an}

%% Experimente
\newcommand{\fopi}{\textit{FOPI}}
\newcommand{\elsa}{\textit{ELSA}}
\newcommand{\cbet}{\texttt{CBELSA/TAPS}}

%%%%%%%%%%%%%%%%%%%%%%%%%%%%%%%%%%%%%%%%%%%%%%%%%%%%%%%%%%%%%%%%%%%%%%%%%%%%%%%%
%% Sonstiges
\newcommand{\ra}{\ensuremath{\Rightarrow}}
\newcommand{\treg}{$^\text{\tiny{\textregistered}}$ }

% Checkboxes
\newcommand{\cbd}{\item[\CheckedBox]}
\newcommand{\cbt}{\item[\Square]}
\newcommand{\cbo}{\item[]}

%% Dingbats
\usepackage{pifont}% http://ctan.org/pkg/pifont
\newcommand{\cmark}{\ding{52}}
\newcommand{\xmark}{\ding{55}}
% http://willbenton.com/wb-images/pifont.pdf 
% https://tex.stackexchange.com/questions/42619/x-mark-to-match-checkmark

%% Bildunterschriften
\newcommand{\mc}[1]{\caption*{\tiny{\color{gray} #1 }}}

%% Standart-Bilder in Beamer
%%  Erzeugt eine column mit Breite #1 (vorher unbedingt \begin{colunms} setzen!)
%%  darin das Bild mit Namen #2 und die Unterschrift #3
\newcommand{\bfig}[3]{
  \column{#1\textwidth}
  \centering
    \begin{figure}
      \includegraphics[width=0.95\textwidth]{#2}
      \mc{#3}
    \end{figure}
}

                         %%
%%                                                                            %%
%% (C) 2018 Philipp Bielefeldt                                                %%
%%%%%%%%%%%%%%%%%%%%%%%%%%%%%%%%%%%%%%%%%%%%%%%%%%%%%%%%%%%%%%%%%%%%%%%%%%%%%%%%

%% Normale Referenzen
\newcommand{\xdir}{\textit{X}-direction}
\newcommand{\ydir}{\textit{Y}-direction}
\newcommand{\zdir}{\textit{Z}-direction}
\newcommand{\xaxis}{\textit{X}-axis}
\newcommand{\yaxis}{\textit{Y}-axis}
\newcommand{\zaxis}{\textit{Z}-axis}
\newcommand{\xyplane}{\textit{XY}-plane}
\newcommand{\xzplane}{\textit{XZ}-plane}
\newcommand{\yzplane}{\textit{YZ}-plane}

%% Programme & Tools
\newcommand{\gfi}{\texttt{G\small{ENFIT}}\,\uppercase\expandafter{\romannumeral 1}}
\newcommand{\gfii}{\texttt{G\small{ENFIT}}\,\uppercase\expandafter{\romannumeral 2}}
\newcommand{\explora}{\texttt{ExPlORA}}
\newcommand{\geant}{\texttt{Geant}}
\newcommand{\minuit}{\texttt{M\small{INUIT}}}

\newcommand{\kalman}{K\'alm\'an}

%% Experimente
\newcommand{\fopi}{\textit{FOPI}}
\newcommand{\elsa}{\textit{ELSA}}
\newcommand{\cbet}{\texttt{CBELSA/TAPS}}

%%%%%%%%%%%%%%%%%%%%%%%%%%%%%%%%%%%%%%%%%%%%%%%%%%%%%%%%%%%%%%%%%%%%%%%%%%%%%%%%
%% Sonstiges
\newcommand{\ra}{\ensuremath{\Rightarrow}}
\newcommand{\treg}{$^\text{\tiny{\textregistered}}$ }

% Checkboxes
\newcommand{\cbd}{\item[\CheckedBox]}
\newcommand{\cbt}{\item[\Square]}
\newcommand{\cbo}{\item[]}

%% Dingbats
\usepackage{pifont}% http://ctan.org/pkg/pifont
\newcommand{\cmark}{\ding{52}}
\newcommand{\xmark}{\ding{55}}
% http://willbenton.com/wb-images/pifont.pdf 
% https://tex.stackexchange.com/questions/42619/x-mark-to-match-checkmark

%% Bildunterschriften
\newcommand{\mc}[1]{\caption*{\tiny{\color{gray} #1 }}}

%% Standart-Bilder in Beamer
%%  Erzeugt eine column mit Breite #1 (vorher unbedingt \begin{colunms} setzen!)
%%  darin das Bild mit Namen #2 und die Unterschrift #3
\newcommand{\bfig}[3]{
  \column{#1\textwidth}
  \centering
    \begin{figure}
      \includegraphics[width=0.95\textwidth]{#2}
      \mc{#3}
    \end{figure}
}

                         %%
%%                                                                            %%
%% (C) 2018 Philipp Bielefeldt                                                %%
%%%%%%%%%%%%%%%%%%%%%%%%%%%%%%%%%%%%%%%%%%%%%%%%%%%%%%%%%%%%%%%%%%%%%%%%%%%%%%%%

%% Normale Referenzen
\newcommand{\xdir}{\textit{X}-direction}
\newcommand{\ydir}{\textit{Y}-direction}
\newcommand{\zdir}{\textit{Z}-direction}
\newcommand{\xaxis}{\textit{X}-axis}
\newcommand{\yaxis}{\textit{Y}-axis}
\newcommand{\zaxis}{\textit{Z}-axis}
\newcommand{\xyplane}{\textit{XY}-plane}
\newcommand{\xzplane}{\textit{XZ}-plane}
\newcommand{\yzplane}{\textit{YZ}-plane}

%% Programme & Tools
\newcommand{\gfi}{\texttt{G\small{ENFIT}}\,\uppercase\expandafter{\romannumeral 1}}
\newcommand{\gfii}{\texttt{G\small{ENFIT}}\,\uppercase\expandafter{\romannumeral 2}}
\newcommand{\explora}{\texttt{ExPlORA}}
\newcommand{\geant}{\texttt{Geant}}
\newcommand{\minuit}{\texttt{M\small{INUIT}}}

\newcommand{\kalman}{K\'alm\'an}

%% Experimente
\newcommand{\fopi}{\textit{FOPI}}
\newcommand{\elsa}{\textit{ELSA}}
\newcommand{\cbet}{\texttt{CBELSA/TAPS}}

%%%%%%%%%%%%%%%%%%%%%%%%%%%%%%%%%%%%%%%%%%%%%%%%%%%%%%%%%%%%%%%%%%%%%%%%%%%%%%%%
%% Sonstiges
\newcommand{\ra}{\ensuremath{\Rightarrow}}
\newcommand{\treg}{$^\text{\tiny{\textregistered}}$ }

% Checkboxes
\newcommand{\cbd}{\item[\CheckedBox]}
\newcommand{\cbt}{\item[\Square]}
\newcommand{\cbo}{\item[]}

%% Dingbats
\usepackage{pifont}% http://ctan.org/pkg/pifont
\newcommand{\cmark}{\ding{52}}
\newcommand{\xmark}{\ding{55}}
% http://willbenton.com/wb-images/pifont.pdf 
% https://tex.stackexchange.com/questions/42619/x-mark-to-match-checkmark

%% Bildunterschriften
\newcommand{\mc}[1]{\caption*{\tiny{\color{gray} #1 }}}

%% Standart-Bilder in Beamer
%%  Erzeugt eine column mit Breite #1 (vorher unbedingt \begin{colunms} setzen!)
%%  darin das Bild mit Namen #2 und die Unterschrift #3
\newcommand{\bfig}[3]{
  \column{#1\textwidth}
  \centering
    \begin{figure}
      \includegraphics[width=0.95\textwidth]{#2}
      \mc{#3}
    \end{figure}
}

