%%%%%%%%%%%%%%%%%%%%%%%%%%%%%%%%%%%%%%%%%%%%%%%%%%%%%%%%%%%%%%%%%%%%%%%%%%%%%%%%
%% Vorlage zur Verwendung mit LaTeX Beamer. Erstellt die Standart-Gruppen-    %%
%% meetings-Präsentation.                                                     %%
%% (C) Philipp Bielefeldt, 2020                                               %%
%%%%%%%%%%%%%%%%%%%%%%%%%%%%%%%%%%%%%%%%%%%%%%%%%%%%%%%%%%%%%%%%%%%%%%%%%%%%%%%%

%% Sollte im Dokument stehen …
% \documentclass[aspectratio=169]{beamer}

%% Input
% \usepackage{etex}
\usepackage[T1]{fontenc}
\usepackage[utf8]{inputenc}
\usepackage{lmodern}
% \usepackage[ngerman]{babel}
\usepackage[british]{babel}


%% Das Template
\usetheme{CambridgeUS}
\usecolortheme{dove}


%% Schaltet die "Weiter" und so Buttons aus
\setbeamertemplate{navigation symbols}{}

%% Keine Seitenzahlen
% \setbeamertemplate{page number in head/foot}{framenumber}

%% Eigener Footer – ohne Gesamtzahl Seiten
\setbeamertemplate{footline} 
{ 
	\leavevmode% 
	\hbox{% 
		\begin{beamercolorbox}[wd=\paperwidth,ht=2.25ex,dp=1ex,center]{footline}% 
			\usebeamerfont{author in head/foot}
			\insertshortauthor 
			\hspace{8em}
			\insertshorttitle
			\hspace{8em}
			\hfill
			% \insertshortdate{}
			\insertframenumber
			% \inserttotalframenumber\hspace*{2ex} 
	\end{beamercolorbox}}% 
	\vskip0pt% 
}
\setbeamercolor{footline}{fg=gray, bg=white}


%% Zeichnungen etc
\usepackage[tight]{units}
% \usepackage{siunitx}
\usepackage{listings}  % Für die Code-Bespiele
\usepackage{tikz} % Für Zeichnungen
%\usetikzlibrary{arrows}
%\usepackage{pstricks}


%% Bilder in typischen Pfaden
\graphicspath{{./img/}{./img/diagrams/}{./img/logos/}{./img/plots/}{./img/screenshots/}{./img/setup/}}


% Mit "appendix" Backups erstellen
\usepackage{appendixnumberbeamer} 
% → nutze
%\appendix
%\section*{Backup}
%\frame{\sectionpage}

%% So kann man sich seine Sectionpages einstellen:
\setbeamertemplate{section page}{
 \begin{center}
  \begin{beamercolorbox}[sep=12pt,center]{part title}
   \usebeamerfont{section title}\insertsection\par
  \end{beamercolorbox}
 \end{center}
}
% → nutze \frame{\sectionpage}
